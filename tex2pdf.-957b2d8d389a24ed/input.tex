\PassOptionsToPackage{unicode=true}{hyperref} % options for packages loaded elsewhere
\PassOptionsToPackage{hyphens}{url}
%
\documentclass[10pt,twoside,spanish,a5paper,]{book}
\usepackage{lmodern}
\usepackage{amssymb,amsmath}
\usepackage{ifxetex,ifluatex}
\usepackage{fixltx2e} % provides \textsubscript
\ifnum 0\ifxetex 1\fi\ifluatex 1\fi=0 % if pdftex
  \usepackage[T1]{fontenc}
  \usepackage[utf8]{inputenc}
  \usepackage{textcomp} % provides euro and other symbols
\else % if luatex or xelatex
  \usepackage{unicode-math}
  \defaultfontfeatures{Ligatures=TeX,Scale=MatchLowercase}
\fi
% use upquote if available, for straight quotes in verbatim environments
\IfFileExists{upquote.sty}{\usepackage{upquote}}{}
% use microtype if available
\IfFileExists{microtype.sty}{%
\usepackage[]{microtype}
\UseMicrotypeSet[protrusion]{basicmath} % disable protrusion for tt fonts
}{}
\IfFileExists{parskip.sty}{%
\usepackage{parskip}
}{% else
\setlength{\parindent}{0pt}
\setlength{\parskip}{6pt plus 2pt minus 1pt}
}
\usepackage{hyperref}
\hypersetup{
            pdftitle={Comité contra la Tortura: Observaciones referidas a las mujeres y las niñas},
            pdfborder={0 0 0},
            breaklinks=true}
\urlstyle{same}  % don't use monospace font for urls
\usepackage[hcentering]{geometry}
\usepackage{graphicx,grffile}
\makeatletter
\def\maxwidth{\ifdim\Gin@nat@width>\linewidth\linewidth\else\Gin@nat@width\fi}
\def\maxheight{\ifdim\Gin@nat@height>\textheight\textheight\else\Gin@nat@height\fi}
\makeatother
% Scale images if necessary, so that they will not overflow the page
% margins by default, and it is still possible to overwrite the defaults
% using explicit options in \includegraphics[width, height, ...]{}
\setkeys{Gin}{width=\maxwidth,height=\maxheight,keepaspectratio}
% Make links footnotes instead of hotlinks:
\DeclareRobustCommand{\href}[2]{#2\footnote{\url{#1}}}
\setlength{\emergencystretch}{3em}  % prevent overfull lines
\providecommand{\tightlist}{%
  \setlength{\itemsep}{0pt}\setlength{\parskip}{0pt}}
\setcounter{secnumdepth}{0}

% set default figure placement to htbp
\makeatletter
\def\fps@figure{htbp}
\makeatother

% This is the LaTeX header, use it to include changes to the default
% Pandoc template.

% Use \copyleft to inject a copyleft symbol (inverted copyright symbol)
\newcommand{\copyleft}{\reflectbox{©}}

% Some useful packages
\usepackage{setspace}
\usepackage{microtype}

% Fancy headers
\usepackage{fancyhdr}
\pagestyle{fancy}
\setlength{\headheight}{15.2pt}
\fancyhead[]{}
\fancyhead[LE]{\footnotesize{\leftmark}}
\fancyhead[RO]{}
\renewcommand{\chaptermark}[1]{\markboth{#1}{}}
\renewcommand{\sectionmark}[1]{\markright{#1}{}}

% Indent paragraphs
\setlength{\parindent}{15pt}

% Make URL break at any point, making long URLs respect the text box.
\expandafter\def\expandafter\UrlBreaks\expandafter{\UrlBreaks%  save the current one
\do\a\do\b\do\c\do\d\do\e\do\f\do\g\do\h\do\i\do\j%
\do\k\do\l\do\m\do\n\do\o\do\p\do\q\do\r\do\s\do\t%
\do\u\do\v\do\w\do\x\do\y\do\z\do\A\do\B\do\C\do\D%
\do\E\do\F\do\G\do\H\do\I\do\J\do\K\do\L\do\M\do\N%
\do\O\do\P\do\Q\do\R\do\S\do\T\do\U\do\V\do\W\do\X%
\do\Y\do\Z}

% Hide image captions
\usepackage[labelformat=empty]{caption}


% This is a hack to make authors appear on the table of contents and
% also as subheadings on chapter titles.
%
% Make article titles in this format:
%
% # Post title -- Author
%
% https://stackoverflow.com/a/1507530
\usepackage{substr}
\usepackage[explicit]{titlesec}
\titleformat{\chapter}[hang]{}{}{0cm}{%
  \Huge \BeforeSubString{ --- }{#1}\\
  {\Large \itshape ---\BehindSubString{ --- }{#1}---}%
}

% Another hack, prevent pandoc from setting the bibliography section
% title as the previous chapter title.
%
% https://tex.stackexchange.com/a/233271
% https://github.com/jgm/pandoc/issues/1632
\titleformat{name=\chapter,numberless}[hang]{}{}{0cm}{%
  \Huge #1\markboth{#1}{#1}%
}
\ifnum 0\ifxetex 1\fi\ifluatex 1\fi=0 % if pdftex
  \usepackage[shorthands=off,main=spanish]{babel}
\else
  % load polyglossia as late as possible as it *could* call bidi if RTL lang (e.g. Hebrew or Arabic)
  \usepackage{polyglossia}
  \setmainlanguage[]{spanish}
\fi

\title{Comité contra la Tortura: Observaciones referidas a las mujeres y las
niñas}
\date{}

\begin{document}
\maketitle

\newpage
\thispagestyle{empty}

\begin{flushleft}\hbox{\Large{Jurisprudencia sobre Derechos Humanos de las Mujeres 2018}}

Comités Monitores de Derechos Humanos de Naciones Unidas
Consejo de Derechos Humanos de las Naciones Unidas
Comisión Interamericana de Derechos Humanos
Mecanismo de Seguimiento de la Convención de Belém do Pará

SISTEMATIZACIÓN DE LAS JURISPRUDENCIAS NACIONALES: DIEGO GUEVARA

Lima, Perú
Jirón Caracas 2624 Jesús María, Lima - Perú. Telefax (511) 463 9237

\vfill
\copyleft  \the\year ISBN XXXXX
\url{https://cladem.org}

© 2018 Comité de América Latina y el Caribe para la Defensa de Derechos de las Mujeres (CLADEM)

\end{flushleft}
\newpage

{
\setcounter{tocdepth}{2}
\tableofcontents
}
\hypertarget{comituxe9-contra-la-tortura-observaciones-referidas-a-las-mujeres-y-las-niuxf1as}{%
\chapter{Comité contra la Tortura: Observaciones referidas a las mujeres
y las
niñas}\label{comituxe9-contra-la-tortura-observaciones-referidas-a-las-mujeres-y-las-niuxf1as}}

\hypertarget{uxba-observaciones-finales-sobre-los-informes-periuxf3dicos-quintoysexto-combinados-de-la-argentina50}{%
\section[1º Observaciones finales sobre los informes periódicos
quinto~y~sexto combinados de la Argentina]{\texorpdfstring{1º
Observaciones finales sobre los informes periódicos quinto~y~sexto
combinados de la Argentina\footnote{CAT/C/ARG/CO/5-6, 24 de mayo de 2017}}{1º Observaciones finales sobre los informes periódicos quinto~y~sexto combinados de la Argentina}}\label{uxba-observaciones-finales-sobre-los-informes-periuxf3dicos-quintoysexto-combinados-de-la-argentina50}}

\begin{enumerate}
\def\labelenumi{\arabic{enumi}.}
\tightlist
\item
  El Comité contra la Tortura examinó los informes periódicos quinto y
  sexto combinados de la Argentina (CAT/C/ARG/5-6) en sus sesiones 1517ª
  y 1520ª (véase CAT/C/SR.1517 y 1520), celebradas los días 26 y 27 de
  abril de 2017, y aprobó las presentes observaciones finales en su
  1537ª sesión, celebrada el 10 de mayo de 2017.
\end{enumerate}

(\ldots{})

\hypertarget{c.-principales-motivos-de-preocupaciuxf3n-y-recomendaciones51}{%
\subsection[C. Principales motivos de preocupación y
recomendaciones]{\texorpdfstring{C. Principales motivos de preocupación
y recomendaciones\footnote{Las recomendaciones se encuentran en negrita.}}{C. Principales motivos de preocupación y recomendaciones}}\label{c.-principales-motivos-de-preocupaciuxf3n-y-recomendaciones51}}

(\ldots{})

\hypertarget{sistema-nacional-de-prevenciuxf3n-de-la-tortura}{%
\subsection{Sistema Nacional de Prevención de la
Tortura}\label{sistema-nacional-de-prevenciuxf3n-de-la-tortura}}

\begin{enumerate}
\def\labelenumi{\arabic{enumi}.}
\setcounter{enumi}{24}
\item
  Si bien aprecia la adopción de la ley que crea el Sistema Nacional de
  Prevención de la Tortura y de su reglamento (véase el párrafo 5, apdo.
  b)~supra), el Comité nota con preocupación que el Comité Nacional para
  la Prevención de la Tortura, encargado de la dirección del Sistema,
  todavía no ha sido creado. Aunque acoge con satisfacción el inicio del
  proceso de selección de los integrantes de dicho comité nacional,
  preocupa al Comité que el nombramiento de seis representantes por
  parte de grupos parlamentarios y uno por el poder ejecutivo, tal y
  como dispone la ley, pueda suscitar conflictos de intereses que
  comprometan su independencia, como ya indicó el Subcomité para la
  Prevención de la Tortura (véase CAT/OP/ARG/1, párr. 16). El Comité
  suscribe asimismo la preocupación del Subcomité para la Prevención de
  la Tortura con respecto al diseño institucional de algunos mecanismos
  locales de prevención, que no cumplirían con los criterios de
  independencia que exige el Protocolo Facultativo de la Convención.
  Preocupa además al Comité que tan solo seis provincias cuenten con
  mecanismos locales que son operativos y algunos se enfrenten a serios
  desafíos presupuestarios para cumplir con su mandato (art. 2).
\item
  \textbf{El Comité urge al Estado parte a avanzar con la conformación
  del Comité Nacional para la Prevención de la Tortura, y velar por que
  sus miembros sean elegidos en un proceso transparente e incluyente ,
  de conformidad con los criterios de independencia, equilibrio de
  género , representatividad de la población, idoneidad y reconocida
  capacidad en diversas áreas multidisciplinarias, incluido en materia
  jurídica y de atención de la salud (véase el artículo 18 del Protocolo
  Facultativo y CAT/ OP /12/5, párrs . 17 a 20). Para ello el Estado
  parte debe abstenerse de nombrar miembros que ocupen cargos que puedan
  suscitar conflictos de intereses (véase CAT/OP/ARG/1, párr. 16). El
  Comité insta asimismo al Estado parte a avanzar en el proceso de
  creación de los mecanismos locales, conforme a los mismos criterios
  arriba citados, y dotarlos de los recursos necesarios para el
  ejercicio de sus funciones.}
\end{enumerate}

(\ldots{})

\hypertarget{violencia-de-guxe9nero-y-mujeres-en-detenciuxf3n}{%
\subsection{Violencia de género y mujeres en
detención}\label{violencia-de-guxe9nero-y-mujeres-en-detenciuxf3n}}

\begin{enumerate}
\def\labelenumi{\arabic{enumi}.}
\setcounter{enumi}{38}
\item
  Si bien acoge con beneplácito las medidas adoptadas para combatir la
  violencia de género (véanse los párrafos 5, apdos. c) y e), y 6, apdo.
  a)~supra) preocupa al Comité el alarmante número de casos de
  femicidios y violencia de género registrados, así como el incremento
  de los casos de violencia física sobre mujeres detenidas bajo
  jurisdicción federal. Aunque aprecia la información proporcionada con
  respecto al seguimiento que se ha dado a los casos de femicidio en
  2015, el Comité lamenta que esta información no se haya extendido al
  período sujeto a examen y a otros casos de violencia de género,
  incluido dentro del sistema penitenciario. El Comité aprecia asimismo
  la información sobre los programas destinados a mejorar el acceso a la
  salud de mujeres en detención, particularmente de mujeres embarazadas;
  no obstante, mantiene su inquietud acerca de la insuficiencia de estos
  programas a nivel federal y provincial, a la vista de las deficiencias
  observadas por diferentes organismos de control (arts. 2, 12 y 16).
\item
  \textbf{El Comité insta al Estado parte a que intensifique sus
  esfuerzos para combatir todas las formas de violencia de género,
  incluido dentro de los centros de privación de libertad, velando por
  que se investiguen a fondo todas las denuncias, se enjuicie a los
  presuntos autores, se les impongan penas apropiadas de ser condenados,
  y velando asimismo por que las víctimas obtengan reparación integral
  del daño. El Comité reitera la recomendación formulada por el Comité
  para la Eliminación de la Discriminación contra la Mujer con respecto
  a las mujeres en detención (véase CEDAW/C/ARG/CO/7, párr. 45) y
  recomienda al Estado parte que desarrolle y mejore los programas de
  acceso a la salud de mujeres en detención a nivel federal y provincial
  (reglas 48 y 51 de las Reglas de Bangkok).}
\end{enumerate}

\end{document}
